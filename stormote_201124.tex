\documentclass[a4paper]{article}

\usepackage{hyperref}
\usepackage{csquotes}
\usepackage{parskip}
\usepackage{titlesec}
\usepackage[swedish]{babel}

\titleformat{\section}{\normalsize\bfseries}{}{0.0em}{}

\begin{document}

\begin{center}
{\huge LiTHe kod höstmöte 2020}\par
\vspace{0.5em}
{\Large 2020--11--24}
\vspace{1.5em}
\end{center}

Mötet öppnas 17:17.

\hypertarget{val-av-muxf6tesordfuxf6rande}{%
\section{Val av mötesordförande}\label{val-av-muxf6tesordfuxf6rande}}

Edvard Thörnros nominerar sig själv. Inga ytterligare kandidater eller
nomineringar. Inga motsättningar. 8 röster för. Mötet beslutar att
Edvard utses till mötesordförande

\hypertarget{val-av-muxf6tessekreterare}{%
\section{Val av mötessekreterare}\label{val-av-muxf6tessekreterare}}

Edvard nominerar Gustav Sörnäs. Inga ytterligare kandidater eller
nomineringar. Inga motsättningar. 8 röster för. Mötet beslutar att
Gustav utses till mötessekreterare.

\hypertarget{val-av-justeringsperson-tillika-ruxf6struxe4knare}{%
\section{Val av justeringsperson tillika
rösträknare}\label{val-av-justeringsperson-tillika-ruxf6struxe4knare}}

Edvard nominerar Annie Wång och Erik Mattfolk. Inga ytterligare kandidater eller
nomineringar. Inga motsättningar. 10 röster för Annie, 8 röster för Erik. Mötet
beslutar att Annie och Erik utses till justeringspersoner tillika rösträknare.

\hypertarget{faststuxe4llande-av-ruxf6stluxe4ngden}{%
\section{Fastställande av
röstlängden}\label{faststuxe4llande-av-ruxf6stluxe4ngden}}

Annie föreslår att röstlängden korrigeras inför varje behandlad punkt.
Inga motsättningar.

\hypertarget{beslut-om-muxf6tets-stadgeenliga-utlysande}{%
\section{Beslut om mötets stadgeenliga
utlysande}\label{beslut-om-muxf6tets-stadgeenliga-utlysande}}

Edvard föreslår att mötet accepterar mötets stadgeenliga utlysande. Inga
motsättningar.

\hypertarget{fuxf6reguxe5ende-verksamhetsuxe5rs-styrelses-verksamhetsberuxe4ttelse}{%
\section{Föregående verksamhetsårs styrelses
verksamhetsberättelse}\label{fuxf6reguxe5ende-verksamhetsuxe5rs-styrelses-verksamhetsberuxe4ttelse}}

Edvard läser upp verksamhetsberättelsen.

\begin{displayquote}
  Skakigt i början, miste en huvudsponsor och hade liten uppslutning på nolle-p
  föreläsningen och programmeringstävlingar. Styrelsen tog fram en ny sida och
  medlemsdatabas men det är inte helt klart. COVID-19 har påverkat mycket, till
  exempel har vi haft digitala påskäggsjakten och mycket mer.  Stärka kopplingar
  mellan kurser och lithekod, IDA har dock blivit bättre och jobbar mer på det
  som behövs, såsom git föreläsningar. Föreningen står sig bra. Finns ny kodapa.
\end{displayquote}

\hypertarget{fuxf6reguxe5ende-verksamhetsuxe5rs-styrelses-ekonomiska-beruxe4ttelse}{%
\section{Föregående verksamhetsårs styrelses ekonomiska
berättelse}\label{fuxf6reguxe5ende-verksamhetsuxe5rs-styrelses-ekonomiska-beruxe4ttelse}}

Edvard läser upp den ekonomiska berättelsen.

\begin{displayquote}
  Blev av med huvudsponsor, gått minus \textasciitilde{}30.000, event blev inte
  av på grund av Corona.
\end{displayquote}

\hypertarget{revisorns-granskning-av-fuxf6reguxe5ende-verksamhetsuxe5rs-styrelses-arbete}{%
\section{Revisorns granskning av föregående verksamhetsårs styrelses
arbete}\label{revisorns-granskning-av-fuxf6reguxe5ende-verksamhetsuxe5rs-styrelses-arbete}}

Edvard lämnar ordet till revisorn, Hugo Hörnquist. Revisorns revidering
följer.

\begin{displayquote}
  Mötesprotokollen finns, dock är de ofta oklara på vad som
  faktiskt har gjorts. Oavsett verkar styret ha utfört sitt
  arbete.

  Så vitt jag kan se har budgeten mer eller mindre följts.
  Föreningen har gått \textasciitilde{}30.000 back, och hade budgeterat för
  \textasciitilde{}40.000 back, vilket får anses vara nära nog. Stora mängder
  pengar har gått till priser samt resor för diverse
  programmeringstävlingar, utan tydlig bokföring vem som fått
  förmånerna. Finns även en suspekt utbetalning den 13
  februari (2020) till Olav Övrebrö om 26.081:- vilken är för
  ``Allt\ldots \ Typ\ldots''.

  Jag misstänker inget fuffens för någon av ovanstående, men
  är värda att ta i beaktande innan ansvarsfrihet beviljas.
\end{displayquote}

Olav Övrebro (föregående verksamhetsårs styrelses ordförande) bemöter den
kommenterade utbetalningen i fråga.

Till framtida styrelser rekommenderar revisorn att bokföringen bör:

\begin{enumerate}
\def\labelenumi{\arabic{enumi}.}
\item
  Undvika vaga kommentarer som ``Allt\ldots{} Typ\ldots{}''
\item
  Vara tydligare med till vem priser har gått så att det kan verifieras.
\end{enumerate}

Revisorn yrkar på ett av två följande alternativ.

\begin{enumerate}
\def\labelenumi{\arabic{enumi}.}
\item
  Bordlägg frågan om ansvarsfrihet till nästa möte i väntan på vidare
  revidering, under förutsättning att revisorn får tillgång till
  kvitton, e-post och dylikt för att kunna utföra en revidering.
\item
  Acceptera ansvarsfrihet.
\end{enumerate}

Olav yrkar på ett bordläggande.

\hypertarget{beslut-om-ansvarsfrihet-av-fuxf6reguxe5ende-verksamhetsuxe5rs-styrelse}{%
\section{Beslut om ansvarsfrihet av föregående verksamhetsårs
styrelse}\label{beslut-om-ansvarsfrihet-av-fuxf6reguxe5ende-verksamhetsuxe5rs-styrelse}}

Styrelsen vars ansvarsfrihet beslutas om lägger ner sina röster.

Mötet nekar ansvarsfrihet med 2/3/1 röster för/emot/blankt. Mötet godtar
att frågan bordläggs till nästa stormöte med 6/1/0 röster
för/emot/blankt.

\hypertarget{motioner-och-propositioner}{%
\section{Motioner och
propositioner}\label{motioner-och-propositioner}}

Inga motioner eller propositioner har skickats in.

\hypertarget{uxf6vriga-fruxe5gor}{%
\section{Övriga frågor}\label{uxf6vriga-fruxe5gor}}

\begin{enumerate}
\def\labelenumi{\arabic{enumi}.}
\item
  Lägg till punkt på föredragningslistan om att besluta om närvarande
  stödmedlemmars rösträtt. Mötet diskuterar och kommer fram till att
  stadgarna bör låtas vara som de är och att styrelsen till framtiden
  kan ha en inofficiell ``mötesguide''.
\end{enumerate}

Inga övriga frågor.

Mötet avslutas 17:55.

\vspace{2em}

\begin{tabular}{@{}p{0.5\textwidth}p{0.4\textwidth}@{}}
  \textbf{Mötesordförande} & \textbf{Mötessekreterare} \\[0.3em]
  Edvard Thörnros & Gustav Sörnäs \\
  \vspace{8em} &\\
  \textbf{Justeringsperson} & \textbf{Justeringsperson} \\[0.3em]
  Annie Wång & Erik Mattfolk \\
  \vspace{8em} &\\
\end{tabular}

\end{document}
