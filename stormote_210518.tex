\documentclass[a4paper]{article}

\usepackage{geometry}
\usepackage{hyperref}
\usepackage{csquotes}
\usepackage{parskip}
\usepackage{titlesec}
\usepackage{enumitem}
\usepackage{caption}
\usepackage[swedish]{babel}

\titleformat{\section}{\large\bfseries}{}{0.0em}{}
\titleformat{\subsection}{\normalsize\bfseries}{}{0.0em}{}

\begin{document}

\begin{center}
{\huge LiTHe kod vårmöte 2021}\par
\vspace{0.5em}
{\Large LiTHe kods discordserver}\par
{\Large 2020\hspace{0.1em}--\hspace{0.1em}05\hspace{0.1em}--\hspace{0.1em}18}
\vspace{1.5em}
\end{center}

\section{Mötets öppnande}

Mötet öppnas 17:17.

\section{Val av mötesordförande}

Edvard Thörnros nominerar sig själv. Inga motsättningar.

\section{Val av mötessekreterare}

Edvard nominerar Gustav Sörnäs. Inga motsättningar.

\section{Val av justeringsperson tillika rösträknare}

Edvard nominerar Annie Wång och Erik Mattfolk. Inga motsättningar.

\section{Fastställande av röstlängden}

Edvard föreslår att Olav Övrebro och Axel Nordanskog som icke-studerande
stödmedlemmar ges rösträtt i enlighet med stadgarna. Inga motsättningar.

Annie föreslår att röstlängden korrigeras och räknas om inför varje fråga. Inga
motsättningar. Röstning sker genom reaktioner på meddelanden som skickas i
möteskanalen.

Vid röstning noteras röstresultatet som a/b/c där a är antalet röster för, b är
antalet röster mot och c är antalet blanka röster. Röstlängden specifieras i
samband med varje röstning.

\section{Beslut om mötets stadgeenliga utlysande}

Ingen motsättningar. En kommentar lyfts om att hemsidan kanske är lite
svårnavigerad.

\section{Föregående verksamhetsårs styrelses verksamhetsberättelse}

Edvard läser upp verksamhetsberättelsen.

\begin{displayquote}
  Under verksamhetsåret 20-21 har LiTHe-kod hållt,
  och försökt hålla, i många event.
  LiU Game Jam testade Micro Jam conceptet, som introducerade
  många nya gamejammers.
  Vi började året med en Nollep-föreläsning med Mindroad.
  Lödölen som planerades med MindRoad var tvungen att ställas in,
  men hade rekordmånga intresseanmälningar.
  Advent of code nådde nya höjder i deltagand med 52
  tävlande och mer än 12000kr insamlade.
  Den årliga Git-föreläsningen blev en enorm succé
  med över 150 deltagande.
  Operas julstuga (LORD) blev en succé, där många stridsvagnar skapades och sprängdes.
  Minecraft-triatlonet samlade LiTHe-kodare från alla världens
  hörn, och många priser gavs ut.
  IMPA började starkt men förlorade lite energi efter hand.
  NCPC gick av stapeln med väldigt få deltagare,
  på grund av distansläget.
  NWERC hade vi ett lag som deltog i.
  GameJams har rullat på och anpassats till distansläget. Uppslutningen har varit bra trots distansläget.
  Swedish Coding Cup ställdes in på grund av kommunikationsmissar.
  Sett från helheten går det bra för föreningen, det är sorgligt att
  covid-19 gjorde det svårt med att få nya medlemmar.
\end{displayquote}

\section{Föregående verksamhetsårs styrelses ekonomiska berättelse}

Victor läser upp den ekonomiska berättelsen.

\begin{displayquote}
  Vi har lägre utgifter än vad budgeten säger att vi borde haft. 
  De utgiftsnivåer som etablerades i föregående budget är ändå relativt
  rimliga och kan fullföljas ifall distansläget hävs. 
  
  Mindroad blev inte huvudsponsor så inkomsten blev lite mindre än vad 
  föregående budget tillskrev. Vi fick även färre nya medlemmar så inkomst 
  från medlemsavgifterna blev också lägre.
  
  \begin{itemize}[leftmargin=*]
    \item LiTHe Kod hade en omsättning på ca 271~700 kr mot de 438~000 kr som förväntades i förgående budget. 
    \item LiThe Kod hade intäkter som summerat blev ca 170~400 kr mot de 208~000 kr som förväntades i förgående budget.
    \item LiThe Kod hade utgifter som summerat blev ca 100~100 kr mot de 230~000 kr som förväntades i förgående budget.
    \item LiTHe Kod hade totalt en vinst på ca 70~000 kr mot en förväntad förlust på 22~000 kr som förväntades i föregående budget.
    \item Nuvarande har LiTHe Kod en balans på ca 204~000 kr som förväntas ligga relativt stabilt tills nästa styrelse år. 
  \end{itemize}
\end{displayquote}

\section{Revisorns granskning av föregående verksamhetsårs styrelses arbete}

Olav Övrebro läser upp sin granskning av styrelsen.

\begin{displayquote}

Det är tydligt att styrelsen ansträngt sig för att nå ut till gamla såväl som
nya medlemmar, även om covid utgjort ett hinder, och att företagskontakter
värnats om. Det bör dock ifrågasättas huruvida all aktivitet som finansierats
varit helt inom ramen för föreningens verksamhets- och intresseområde, med
exempelvis donationer till WHO (i anknytning till Advent of Code) och prispengar
i en spelturnering (Minecraft Triathlon).

I budgeten uppfattas ett par oegentligheter:
\begin{enumerate}
\item Det finns en oförklarad intäkt om 642 kronor, 19:e januari, endast märkt ??.
\item En planerad utgift (styrelsens team-building; laser dome) har uteblivit helt.
\item Kontoavgifterna uppfattas av revisorn som väldigt höga (1277:-); har föreningen accepterat en dålig deal från banken?
\end{enumerate}

Bortsätt från punkterna ovan uppfattas kassörens arbete under året som bra och
noggrannt. Det sticker såklart ut att föreningen gått så pass mycket plus, och
det föreslås att styrelsen kollar på vid-behov-utgifter att ta till om
föreningen fortsätter att gå med oväntat mycket vinst, och att
budget/verksamhetsplan förbereds för såväl ifall covid-restriktioner fortsätter
gälla, såväl som om de upphör.

\end{displayquote}

De numrerade punkterna ovan diskuteras separat enligt nedan.

\begin{enumerate}
  \item Detta kan röra sig om en ränteintäkt från banken.
  \item Detta uteblev och ersattes med en betydligt billigare filmkväll.
  \item Detta kan vara en kombination av swish-kostnader och högre kostnader för
  föreningar, samt att revisorn är van vid förmånskundspriser hos bank.
\end{enumerate}

Angående WHO lyfts Ingenjörer utan gränser som en organisation som kanske passar
föreningen bättre. Datatjej nämns också men ska enligt uppgift inte ha något
enkelt sätt att donera pengar.

Styrelsen nämner att spelturneringen var ett engångsevenemang i slutet av
terminen som de jämför med en avslutningsmeetup som, om det hade varit på plats,
också hade varit lite mer påkostad. Olav nämner också att prispengarna i
budgeten är markerade som ``Styrelsen till förfogande'' och att styrelsen enligt
stadgarna har stor frihet med hur de pengarna spenderas.

Stormötet såg inte punkterna ovan som större problem utan håller med Olav om att de
mer kan ses som kommentarer till framtida styrelser.

\section{Beslut om ansvarsfrihet av föregående verksamhetsårs styrelse (från höstmötet 2020)}

Uppskjuten punkt från höstmötet 2020.

Revisorn för föregående verksamhets års styrelse Hugo Hörnquist är inte
närvarande men har skickat ett utlåtande via mail. Edvard läser upp utlåtandet.

\begin{displayquote}
  Efter att ha kollat igenom dokumenten ser jag anser jag att allting
  ser bra ut, och ser ingen anledning till att inte bevilja den
  styrelsen ansvarsfrihet.
\end{displayquote}

Eftersom Hugo har varit svår att nå beslutade styrelsen att be Frans Skarman
också granska materialet den 2020--05--13. Frans läser upp sitt utlåtande.

\begin{displayquote}
  Jag har granskat den ekonomiska biten och även om jag hittade några fel så har de nu åtgärdats.

  Jag ser ingen anledning att inte bevilja ansvarsfrihet till styrelsen 19/20.
\end{displayquote}

Stormötet beslutar att bevilja anvarsfrihet för föregående års styrelse med
rösttal 8/0/6, röstlängd 14. Styrelsen vars ansvarsfrihet röstades om lade ner
sina röster.

\section{Val av ordförande 21/22}

Edvard nominerar Gustav Sörnäs. Gustav presenterar sig själv. Frågor ställs. Diskussion
hålls med Gustav dämpad.

Stormötet beslutar att välja Gustav som ordförande för 21/22 med rösttal 12/0/0, röstlängd 12.

\section{Val av vice ordförande 21/22}

Gustav presenterar Lowe Kozaw Åslöw. Lowe presenterar sig själv. Frågor ställs.
Diskussion hålls med Lowe dämpad.

Stormötet beslutar att välja Lowe som vice ordförande för 21/22 med rösttal
13/0/0, röstlängd 13.

\section{Val av kassör 21/22}

Gustav nominerar Victor Lells. Victor presenterar sig själv. Frågor ställs. Diskussion
hålls med Victor dämpad.

Stormötet beslutar att välja Victor som kassör för 21/22 med rösttal 13/0/0, röstlängd 13.

\section{Val av verksamhetsansvarig 21/22}

Gustav nominerar Axel Telenius. Axel presenterar sig själv. Frågor ställs. Diskussion
hålls med Axel dämpad.

Stormötet beslutar att välja Axel som verksamhetsansvarig för 21/22 med rösttal 13/0/0, röstlängd 13.

\section{Val av PR-ansvarig 21/22}

Ingen motsätter sig att vakantsätta rollen. Ansvarsområdet har under året
spridits ut på styrelsen. Det nämns att en PR-ansvarig som vet vad den håller på
med kan vara till stor hjälp. Tyvärr inkom det inte några fler sökande.

\section{Val av webbansvarig 21/22}

Gustav nominerar Erik Mattfolk. Erik presenterar sig själv. Frågor ställs. Diskussion
hålls med Erik dämpad.

Stormötet beslutar att välja Erik som webbansvarig för 21/22 med rösttal 12/0/0, röstlängd 12.

\section{Val av Game Jam-ansvarig 21/22}

Gustav nominerar Thea Carlqvist. Thea presenterar sig själv. Frågor ställs. Diskussion
hålls med Thea dämpad.

Stormötet beslutar att välja Thea som game jam-ansvarig för 21/22 med rösttal 12/0/0, röstlängd 12.

\section{Val av revisor 21/22}

Edvard Thörnros nominerar sig själv. Edvard presenterar sig själv. Frågor ställs. Diskussion
hålls med Edvard dämpad.

Stormötet beslutar att välja Edvard som revisor för 21/22 med rösttal 12/0/0, röstlängd 12.

\section{Fastställande av föreningens budget 21/22}

Victor presenterar ett budgetförslag. Enda skillnaden mot föregående år är att
Mindroad inte längre är huvudsponsor.

Stormötet beslutar att anta en svagt modifierad budget (tabell~\ref{tab:budget}) utan motsättningar.

\begin{table}[ht]
  \centering
  \caption{Antagen budget för verksamhetsåret 21/22}\label{tab:budget}
  \begin{tabular}{l r}
    Kategori & Budget \\
    \hline
    \\
    Intäkter \\
    Ericsson & 40 000 kr \\
    IDA Infront & 40 000 kr \\
    Opera & 40 000 kr \\
    Axis & 40 000 kr \\
    Mindroad & 20 000 kr \\
    Medlemsavgifter & 1 000 kr \\
    Styrelsetröjor & 10 000 kr \\
    \hline
    Totala intäkter & 191 000 kr \\
    \\
    Utgifter \\ 
    Styrelsen till förfogande & 45 000 kr \\
    NCPC & 35 000 kr \\
    IMPA & 35 000 kr \\
    Game jam-gruppen & 45 000 kr \\
    Events & 45 000 kr \\
    Bidragsfond & 25 000 kr \\
    \hline
    Totala utgifter & 230 000 kr \\
    \\
    Totalt & -39 000 kr \\
    Omsättning & 421 000 kr
  \end{tabular}
\end{table}

\section{Fastställande av medlemsavgift 21/22}

Stormötet beslutar att inte ändra den årliga medlemsavgiften från nuvarande 20 kr för icke-medlemmar
och 0 kr för medlemmar.

\section{Motioner och propositioner}

Inga motioner eller propositioner har skickats in.

\section{Övriga frågor}

\begin{enumerate}
\def\labelenumi{\arabic{enumi}.}
\item Stormötet beslutar att verksamhetsledare (som det benämns i stadgarna)
och verksamhetsansvarig (som det benämnts på mötet) ska anses som samma roll.
\end{enumerate}

Inga övriga frågor.

Mötet avslutas 19:17.

\vspace{2em}

\begin{tabular}{@{}p{0.5\textwidth}p{0.4\textwidth}@{}}
  \textbf{Mötesordförande} & \textbf{Mötessekreterare} \\[0.3em]
  Edvard Thörnros & Gustav Sörnäs \\
  \vspace{8em} &\\
  \textbf{Justeringsperson} & \textbf{Justeringsperson} \\[0.3em]
  Annie Wång & Erik Mattfolk \\
  \vspace{8em} &\\
\end{tabular}

\end{document}
