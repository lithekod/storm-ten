\documentclass[a4paper]{article}
\usepackage[T1]{fontenc}

\usepackage{geometry}
\usepackage{hyperref}
\usepackage{csquotes}
\usepackage{parskip}
\usepackage{titlesec}
\usepackage{enumitem}
\usepackage{caption}
\usepackage[swedish]{babel}
\usepackage{framed}

\titleformat{\section}{\large\bfseries}{}{0.0em}{}
\titleformat{\subsection}{\normalsize\bfseries}{}{0.0em}{}

\newenvironment{quotationb}%
{\begin{leftbar}}%
{\end{leftbar}}

\begin{document}

\begin{center}
{\huge LiTHe kod höstmöte 2022}\par
\vspace{0.5em}
{\Large Ada Lovelace}\par
{\Large 2022\hspace{0.1em}--\hspace{0.1em}11\hspace{0.1em}--\hspace{0.1em}22}
\vspace{1.5em}
\end{center}

\section{Mötets öppnande}

Mötet öppnas 18:15.

\section{Val av mötesordförande}

Lowe Kozak nominerar sig själv. Inga motsättningar.

\section{Val av mötessekreterare}

Lowe nominerar Emanuel Särnhammar. Emanuel vill inte vara sekreterare. Gustav
Sörnäs nominerar sig själv.

Mötet beslutar att välja Gustav till mötessekreterare.

\section{Val av justeringsperson tillika rösträknare}

Lowe nominerar Hannah Bahremann och Frans Skarman. Inga motsättningar.

\section{Fastställande av röstlängden}

Röstlängden fastställs till 13.

\section{Beslut om mötets stadgeenliga utlysande}

Ingen motsättningar.

\section{Föregånde verksamhetsårs styrelses verksamhetsberättelse}

Gustav läser upp verksamhetsberättelsen.

\begin{quotationb}

Verksamhetsåret började med en stor osäkerhet kring COVID. Efter en stund kom vi
    igång och numera har vi även styrelsemöten på plats. Om inte allt går helt
    sidleds under sommaren ser vi fram emot ett helt verksamhetsår på plats
    nästa år.

På inbjudan av Köpenhamns universitet skickade vi två lag till Danmark under
    NWERC. sigma bools placerade sig på 97:e plats med 3 lösta problem. Nästa år
    ser vi fram emot att skicka iväg några lag ut i Europa.

Admin-mässigt har vi rensat databasen på inaktiva medlemmar. Numera har vi 93
    aktiva medlemmar. Anekdotiskt har vi sett några nya ansikten, och den
    föreslagna styrelsen är till och med fler nya än gamla!

På höstmötet fick styrelsen i uppdrag att se över protokollföringen så att den
    följer GDPR. Resultatet av arbetet presenteras i form av en proposition
    senare under mötet.

På game jam-fronten gick Fall game jam på plats med bra närvaro i Ebbepark efter
    att Creactive fortsatt inte öppnat. Kanske kommer vi kunna återvända dit i
    höst efter att Internetstiftelsen tagit över. Tagga tagga! Global game jam
    gick på distans med närvaron som hör till det, och Spring har i skrivande
    stund inte skett. Game jam-gruppen testade att stå i Collo (föreningens
    första?) med oklara resultat. Folk hade kul så det kommer nog fortsatt ske.

\end{quotationb}

\section{Föregånde verksamhetsårs styrelses ekomomiska berättelse}

Victor Lells läser upp den ekonomiska berättelsen.

\begin{quotationb}

\textbf{Utgångspunkt 2021/2022}

LiTHe Kod började året med 190,099.73 sek på kontot efter att ha
misslyckats spendera pengar år 2020/2022.

\textbf{Inkomster 2021/2022}

Styrelseår 2021/2022 har LiTHe Kod haft fyra huvudsponsorer (Axis,
Ericsson, IDA Infront, och Opera) som alla gav 40'000 sek för att vara
synliga på IMPA och NCPC. LiTHe Kod hade även ett separat sponsoravtal
med Mindroad som för diverese event och förmåner betalade 20'000
sek. Opera betalade även 5'000 sek för att synas på styrelsehoodies.

Utöver detta har vi fått 540 sek i medlemsavgifter och 20'000 från
Mindorad som inte betalades in för styrelseår 2020/2021 förrän detta
budgetår påbörjades. Total Inkomst för 2021/2022 är 205'540 sek detta
är i jämförelse med att vi budgeterade en inkomst på 191,000.00 sek.

\textbf{Utgifter 2021/2022}

Styrelseår 2021/2022 har LiTHe kod spenderat pengar på IMPA,
NCPC/ICPC, Event(Meetup), Mottagningen(Nolle-p), Marknadsföring och
GameJams. Några av budgetposterna har överstigit gränserna som
etablerades i budgeten för styrelse år 2021/2022 (Event och Övriga
Utgifter) medan andra ligger långt under gränsen (IMPA/NCPC, GameJam,
och Stöd för Programmeringsrelaterade ändamål).

 Nuvarande kassör förväntar sig att 10'000-20'000 sek kommer spenderas
på ett sista Game Jam samt att sista omgången av IMPA kommer resultera
i 10'000 sek i prispengar för deltagarna.

Totala Utgifter för styrelseår 2021/2022 är för nuvarande 172,129.43
kr sek och förväntas stiga till ca 200,000.00 sek innan slutet av
styrelseåret.

\textbf{Tillbakablick 2021/2022}

LiTHe kod (exklusive tidigare års inkomster) förväntar sig en förlust
på ca 20'000 sek ifall planerade utgifter fullföljs. Men i nuläget
har LiTHe kod 219,775.52 sek på kontot och har därmed i nuläget gått
med en vinst på 9'675.79 sek.

\end{quotationb}

Röstlängden justeras till 14.

Siffrorna kontrolleras under mötet eftersom verksamhetsåret numera är fullföljt.
Resultatet blev istället en förlust om 16'787,47 kr.

\section{Revisorns granskning av föregående verksamhetsårs styrelses arbete}

Granskningen läses upp. Revisorn tillstyrker att föregående verksamhetsårs
styrelse beviljas ansvarsfrihet.

\begin{quotationb}

''Bokföringen'' ser ut att ha sköts mycket bra. Revisorn anser att styrelsen bör
    få ansvarsfrihet.

\end{quotationb}

\section{Beslut om ansvarsfrihet av föregående verksamhetsårs styrelse}

Mötet beslutar att bevilja ansvarsfrihet till föregående verksamhetsårs
styrelse.

\section{Motioner och propositioner}

Inga inkomna.

\section{Övriga frågor}

\begin{enumerate}
\def\labelenumi{\arabic{enumi}.}
\item \emph{Fråga:} När kommer pizzan? \\
    \emph{Svar:} Om 15-20 minuter.
\end{enumerate}

Mötet avslutas 18:34.

\vspace{2em}

\begin{tabular}{@{}p{0.5\textwidth}p{0.4\textwidth}@{}}
  \textbf{Mötesordförande} & \textbf{Mötessekreterare} \\[0.3em]
  Lowe Kozak & Gustav Sörnäs \\
  \vspace{8em} &\\
  \textbf{Justeringsperson} & \textbf{Justeringsperson} \\[0.3em]
  Hannah Bahremann & Frans Skarman \\
  \vspace{8em} &\\
\end{tabular}

\end{document}
